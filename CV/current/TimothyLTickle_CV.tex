\documentclass[12pt]{report}
\usepackage{multirow}%Allows you to span multiple rows
\usepackage{fullpage}%Sets pages to 1in margins
\usepackage[resetlabels]{multibib}%Allows multiple bibliographies
\usepackage[margin=1.0in]{geometry}%Allows setting space
\usepackage{setspace}%To create custome spacing in a setspacing environment
\usepackage{fancyhdr}%To add in custom headers and footers
\usepackage{lastpage}%To add in custom page numbers the variable \LastPage
\usepackage[dvipdfm]{hyperref}%To add hrefs

%Set margins to 1 inch all around
%\addtolength{\oddsidemargin}{-.875in}
%\addtolength{\evensidemargin}{-.875in}
%\addtolength{\textwidth}{1.75in}
%\addtolength{\topmargin}{-.875in}
%\addtolength{\textheight}{1.75in}

\makeatletter
\renewcommand\@biblabel[1]{}
\renewenvironment{thebibliography}[1]
     {\section*{\refname}%
      \@mkboth{\MakeUppercase\refname}{\MakeUppercase\refname}%
      \list{\@biblabel{\@arabic\c@enumiv}}%
           {\settowidth\labelwidth{\@biblabel{#1}}%
            \leftmargin\labelwidth
            \advance\leftmargin\labelsep
            \itemindent-\labelsep
            \@openbib@code
            \usecounter{enumiv}%
            \let\p@enumiv\@empty
            \renewcommand\theenumiv{\@arabic\c@enumiv}}%
      \sloppy
      \clubpenalty4000
      \@clubpenalty \clubpenalty
      \widowpenalty4000%
      \sfcode`\.\@m}
     {\def\@noitemerr
       {\@latex@warning{Empty `thebibliography' environment}}%
      \endlist}
\makeatother

%Control page layout
\setlength{\topmargin}{-5.0mm}%Gap above header
\setlength{\topskip}{0.0mm}%Gap between header and text

%Control paragraphs
\setlength{\parindent}{0.0mm}%Indention of paragraphs
\setlength{\parskip}{-5.0mm}%Gap between paragraphs

%Control Floats
\setlength{\floatsep}{0.0mm}
\setlength{\textfloatsep}{0.0mm}
\setlength{\intextsep}{2.0mm}
\setlength{\abovecaptionskip}{0.0mm}
\setlength{\belowcaptionskip}{0.0mm}

%Control lists
\addtolength{\topsep}{0.0mm}%Space between first item and preceding paragraph
\addtolength{\partopsep}{0.0mm}%Extra space added to \topsep when environment starts a new paragraph
\addtolength{\itemsep}{0.0mm}%space between list items

%Remove hypenating words
\hyphenpenalty=5000
\tolerance=1000

%Make sections not numbered and formatted
\newcommand{\header}{\section*}
%Make itemized more compact
\newenvironment{detailsList}
{\begin{itemize}
  \setlength{\itemsep}{1pt}
  \setlength{\parskip}{0pt}
  \setlength{\parsep}{0pt}
  \setlength{\partopsep}{0pt}
  \setlength{\topsep}{0pt}}
{\end{itemize}}

%%%Global Variables
\def\yearColumnLength{1.0in}
\def\detailColumnLength{5.5in}
\def\fullLength{6.5in}

%Multiple bibliographies
\newcites{Conf,Abstract,Art}{{\Large Peer Reviewed Publications:},{\Large Conference Posters and Abstracts:},{\Large Literary Works:}}

\renewcommand{\headrulewidth}{0.0pt}%Do not let fancyhdr add a header line
\renewcommand{\footrulewidth}{0.0pt}%Do not let fancyhdr add a footer line

%Add footer
\cfoot{Timothy Tickle \thepage\ of \pageref{LastPage}}

\begin{document}
\pagestyle{fancy}
\fancyhead{}
\begin{table}[!ht]
\begin{tabular}{p{\fullLength}}
\textbf{\Huge Timothy L. Tickle, Ph.D.}\hfill 1-704-777-4245\\
http://TIMOTHYLTICKLE.BITBUCKET.ORG \hfill TIMOTHYLTICKLE@GMAIL.COM\\\hline\hline
\end{tabular}
\end{table}

\vspace{-5.0mm}

%Education
\begin{table}[!ht]
\begin{tabular}{p{\fullLength}}
\textbf{\Large Education:}\\
Harvard School of Public Health (Department of Biostatistics)\hfill September 2011-Present\\
Postdoctoral Fellow (Curtis Huttenhower Lab)\hfill Boston, MA\\
\\
University of North Carolina at Charlotte\hfill May 2011\\
Doctor of Philosophy in Bioinformatics and Computational Biology\hfill Charlotte, NC\\
Dissertation: ``Data Mining the Serous Ovarian Tumor Transcriptome''.\\
Topic included: Ovarian Tumor Genetics, Biomarker Discovery, Transcript-level Analysis.\\
\\
University of North Carolina at Charlotte\hfill August 2004\\
Bachelor of Science: Computer Science\hfill Charlotte, NC\\
Research Project: Design of bioinformatics software for Ovarian Cancer Analysis.\\
\\
Wake Forest University\hfill May 1999\\
Bachelor of Science: History, Education\hfill Winston-Salem, NC\\
\end{tabular}
\end{table}

\begin{table}[!ht]
\begin{tabular}{p{\fullLength}}
\textbf{\Large Research Interests:}\\
Experienced in applying high-throughput technology to complex diseases. Interested in both designing and implementing analysis infrastructure and software systems for bioinformatics analysis. Focused on applying next-generation sequencing, and microarray chip technology  to basic and translational experimentation, explicitly when related to clinical studies.
\end{tabular}
\end{table}

\vspace{-12.0mm}

%Full Manuscripts
\section*{ }
\begin{spacing}{0.9}
\bibliographystyleConf{ieeetr}
\nociteConf{2013_microPITA}
\nociteConf{2013MSBReview}
\nociteConf{2012_IBD_Paper}
\nociteConf{Dissertation}
\nociteConf{HOX_ERBB}
\nociteConf{Null_PValue}
\bibliographyConf{TimothyTickle}
\end{spacing}

\clearpage

%Research Experience
\begin{table}[!ht]
\begin{tabular}{p{\fullLength}}
\textbf{\Large Research Experience:}\\
\textbf{Postdoctoral Fellow}\hfill Fall 2011-Present\\
Department of Biostatistics (Harvard University, Curtis Huttenhower Lab)\hfill Boston, MA\\
\vspace{-7.0mm}
\begin{itemize}\addtolength{\itemsep}{-0.5\baselineskip}
\item Responsible for the development and validation of methodology and software for translating metagenomics studies to human diseases.
\item Assist in and lead clinical study analysis in microbial biomarker associations with genetic, environmental, and host phenotypes.
\item Responsible for adhering to software development practices including versioning, regression/unit testing, lab coding standards, and code reviews.
\item Provide documentation, establish a web presence, and support software users.
\item Write scientific reports and create custom visualizations.
\item Mentor students, lab members, and visiting scientists on analysis and tool use.
\end{itemize}
\end{tabular}
\end{table}

\vspace{-5.0mm}

\begin{table}[!ht]
\begin{tabular}{p{\fullLength}}
\textbf{Research Assistant}\hfill Fall 2007-2011\\
College of Computing and Informatics (UNC-Charlotte)\hfill Charlotte, NC\\
\vspace{-7.0mm}
\begin{itemize}\addtolength{\itemsep}{-0.5\baselineskip}
\item Was responsible for all work associated with the ovarian exon tumor transcriptome study. Solely performed all wet-lab protocols and dry-lab analysis.
\item Developed and performed wet lab protocols including: sample preparation and storage; tissue staining; pathology evaluation; cryosectioning; laser capture microdissection; immunohistochemistry; RNA extraction, and isolation; and cDNA generation, amplification, and preparation for Affymetrix GeneChip Human Exon 1.0 ST Arrays.
\item Participated in training other team members and interns in various wet lab protocols.
\item Performed associated analysis and developed software and database solutions.
\end{itemize}
\end{tabular}
\end{table}

\vspace{-5.0mm}

%Academic Work experience
\begin{table}[!h]
\begin{tabular}{p{\fullLength}}
\textbf{Instructor}\hfill Fall 2010\\
Introduction to Bioinformatics (UNC-Charlotte)\hfill Charlotte, NC\\
\vspace{-7.0mm}
\begin{itemize}\addtolength{\itemsep}{-0.5\baselineskip}
\item Acted as the instructor of record. Was solely responsible for designing and teaching the first undergraduate introductory bioinformatics class.
\item Designed and delivered all lectures and associated presentations.
\item Developed assessment activities including testing, review sessions, labs, and homework.
\end{itemize}
\end{tabular}
\end{table}

\vspace{-5.0mm}

\begin{table}[!h]
\begin{tabular}{p{\fullLength}}
\textbf{Teaching Assistant}\hfill Fall 2004-Fall 2007\\
College of Computing and Informatics (UNC-Charlotte)\hfill Charlotte, NC\\
ITCS 3050 (Intro to Bioinformatics), ITCS 6160 (Programming for Biologists), ITCS 2181(Computer Logic \& Design), ITCS 3183 (Hardware Systems Design), ITCS 3181/5141 (Computer Organization \& Architecture) and ITCS 3650/3651 (Senior Projects).\\
\vspace{-7.0mm}
\begin{itemize}\addtolength{\itemsep}{-0.5\baselineskip}
\item Taught classes on programming and class related software.
\item Held lab sessions, office hours, and weekly tutoring sessions.
\end{itemize}
\end{tabular}
\end{table}

\clearpage

%Honors and Awards
\begin{table}[!h]
\begin{tabular}{p{\fullLength}}
\textbf{\Large Honors, Awards, and Assistantships:}\\
%Add in TA years
International Society for Computational Biology Travel Fellowship\hfill 2012 and 2013\\
GAANN Scholars Fellowship\hfill 2009-2011\\
UNC-Charlotte Biotechnology Conference Finalist\hfill 2010\\
TA of the year for the College of Computing and Informatics\hfill 2008\\
Received the Essam El-Kwae award for student-faculty research\hfill 2005\\
%Reward received for ``Ovarian Cancer Research Using Microarray Analysis''\\
Graduated \emph{Cum Laude} from the University of North Carolina at Charlotte\hfill 2004\\
Graduated \emph{Cum Laude} from Wake Forest University\hfill 1999\\
\end{tabular}
\end{table}

%Presentations
\begin{table}[!h]
\begin{tabular}{p{\fullLength}}
\textbf{\Large Invited Presentations (recent):}\\
Microbial biomaker inference in metadata rich studies.\hfill November 2013\\
(Genomics Retreat, Dana-Farber Cancer Institute)\\
Metagenomic inference and biomarker discovery for the gut microbiome\hfill July 2013\\in inflammatory bowel disease.\\
(International Conference on Intelligence Systems for Molecular Biology)\\
Surveying 16S rRNA abundance to determine sample selection\hfill Feb 2012\\
for follow-up studies.\\
(Computational Genomics Group, Dana-Faber Cancer Institute)\\
%CMC presentation
Ovarian Serous Type I,II Tumor Data Set for Expression and\hfill Mar 2011\\
Paracrine Signaling Investigation.\\
(Carolinas Medical Center, Cannon Research Community)\\
Data Mining the Ovarian Tumor Transcriptome.\hfill Feb 2011\\
(UNC-Charlotte's Department of Bioinformatics and Genomics)\\
Harnessing the Secrets of Life: How Bioinformatics is Changing Our World.\hfill Apr,~Mar 2010\\
(Queens University and Wingate University)\\
%Presented Multiple Presentations on Microarray Technology.\hfill May-Dec 2005\\
%Automated Microarray Analysis\hfill Dec 2005\\
%(UNC-Charlotte's Bioinformatics Research Group)\\
%Introduction to Bioinformatics and Microarray Technology\hfill Nov 2005\\
%(UNC-Charlotte's Computer Science Freshman Community)\\
%Ovarian Cancer Microarray Analysis\hfill Nov 2005\\
%(UNC-Charlotte's Computer Science Research Seminar)\\
%Analysis of Ovarian Microarray Data\hfill Apr 2005\\
%(UNC-Charlotte's Graduate Research Fair)\\
%Bioinformatics: A Study of Ovarian Cancer.\hfill May 2004\\
%(Carolina's Medical Center, Molecular Biology Core Facility)\\
\end{tabular}
\end{table}

%Guest Lecturing
\begin{table}[!h]
\begin{tabular}{p{\fullLength}}
\textbf{\Large Guest Lecturing:}\\
Mentor Graphics Tutorial\hfill Sept 2010\\
(Computer Systems Lab and Recitation; UNC-Charlotte)\hfill \\
Microarray Technology\hfill Apr 2010\\
(Bioinformatics; Davidson College)\hfill \\
Unit Testing in Python\hfill Mar 2010\\
(Bionformatics Progamming I; UNC-Charlotte)\hfill \\
Mentor Graphics Tutorial\hfill Sept 2009\\
(Computer Architecture/Hardware Design; UNC-Charlotte)\hfill \\
Unit Testing in Python\hfill Nov 2009\\
(Bioinformatics Programming I; UNC-Charlotte)\hfill \\
\end{tabular}
\end{table}

%Data sets
\begin{table}[!h]
\begin{tabular}{p{\fullLength}}
\textbf{\Large Public Data Sets Developed:}\\
Serous ovarian benign tumor and type II carcinoma data set for expression and paracrine signaling investigation (GEO\#~GSE29156).
\end{tabular}
\end{table}

%Professional Societies
\begin{table}[!h]
\begin{tabular}{p{\fullLength}}
\textbf{\Large Professional Societies:}\\
The International Society for Computational Biology\hfill 2010-Present\\
American Association of Cancer Researchers\hfill 2004-2007\\
\end{tabular}
\end{table}

%Technical Skills
\begin{table}[!ht]
\begin{tabular}{p{\fullLength}}
\textbf{\Large Computer Skills:}\\
\textbf{Operating Systems: }Windows~(current), Mac~OS~(desktop~and~server), Linux~(Ubuntu)\\
\textbf{Programming and Scripting Related: }C/C++/C\#, Java~(SE,~EE), JDBC,\\
Java~2D~API, JSP, JUnit, XML, JSON, Python, NumPy, mlpy, matplotlib, PyUnit, PyCogent, Biopython, R\\
\textbf{Bioinformatics Related: }16S~amplicon and whole~microbial~metagenomics analysis,\\ BLAST, Bowtie2, metaPhlan, HUMAnN, LefSe, and other Huttenhower tools\\
\textbf{Databases: }Oracle, Postgres, SQL\\
\textbf{Other Applications: }Bitbucket / mercurial, Eclipse, Inkscape, \LaTeX, bib\TeX\\
\textbf{Algorithms: }Support Vector Machines (SVMs), Gradient Boosting,\\
Multivariate Regression, K-mediods, Multiple Factor Analysis, Principle Components Analysis (PCA), Principle Coordinates Anlysis (PCoA), Nonmetric Multidimensional Scaling (NMDS), Hierarchical Clustering\\
 \\
\textbf{\Large Wet Lab Skills:}\\
\textbf{Antibody Based: } Immunohistochemistry\\
\textbf{Laser Capture Microdissection: }Manual (PixCell IIe) and Automated (ArcturusXT\textsuperscript{\texttrademark}); cut and capture, cryosectioning\\
\textbf{General Molecular Biology Techniques: }Purification columns, affynity beads, RNA (extraction, isolation), cDNA (isolation, amplification), Nanodrop, Bioanalyzer\\
\textbf{Human Study Specific: }Human tissue and fluid (serum) preparation for tissue bank, familiar with ovarian tumor pathology\\
\textbf{Next Generation Sequencing:} cDNA library construction, emulsion PCR (emPCR)\\
\textbf{Microarray Related: }ST-cDNA conversion, cDNA fragmentation and labeling, sample preparation for Affymetrix GeneChip Human Exon 1.0 ST Arrays using NuGEN products\\
\textbf{Staining: }Hemotoxylin and eosin, HistoGene\\
\end{tabular}
\end{table}
\end{document}
